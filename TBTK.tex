\PassOptionsToPackage{unicode=true}{hyperref} % options for packages loaded elsewhere
\PassOptionsToPackage{hyphens}{url}
%
\documentclass[
  12pt,
  ignorenonframetext,
]{beamer}
\usepackage{pgfpages}
\setbeamertemplate{caption}[numbered]
\setbeamertemplate{caption label separator}{: }
\setbeamercolor{caption name}{fg=normal text.fg}
\beamertemplatenavigationsymbolsempty
% Prevent slide breaks in the middle of a paragraph:
\widowpenalties 1 10000
\raggedbottom
\setbeamertemplate{part page}{
  \centering
  \begin{beamercolorbox}[sep=16pt,center]{part title}
    \usebeamerfont{part title}\insertpart\par
  \end{beamercolorbox}
}
\setbeamertemplate{section page}{
  \centering
  \begin{beamercolorbox}[sep=12pt,center]{part title}
    \usebeamerfont{section title}\insertsection\par
  \end{beamercolorbox}
}
\setbeamertemplate{subsection page}{
  \centering
  \begin{beamercolorbox}[sep=8pt,center]{part title}
    \usebeamerfont{subsection title}\insertsubsection\par
  \end{beamercolorbox}
}
\AtBeginPart{
  \frame{\partpage}
}
\AtBeginSection{
  \ifbibliography
  \else
    \frame{\sectionpage}
  \fi
}
\AtBeginSubsection{
  \frame{\subsectionpage}
}
\usepackage{lmodern}
\usepackage{amssymb,amsmath}
\usepackage{ifxetex,ifluatex}
\ifnum 0\ifxetex 1\fi\ifluatex 1\fi=0 % if pdftex
  \usepackage[T1]{fontenc}
  \usepackage[utf8]{inputenc}
  \usepackage{textcomp} % provides euro and other symbols
\else % if luatex or xelatex
  \usepackage{unicode-math}
  \defaultfontfeatures{Scale=MatchLowercase}
  \defaultfontfeatures[\rmfamily]{Ligatures=TeX,Scale=1}
  \setmainfont[]{Times New Roman}
\fi
\usetheme[]{dresden}
\usefonttheme{serif} % use mainfont rather than sansfont for slide text
% use upquote if available, for straight quotes in verbatim environments
\IfFileExists{upquote.sty}{\usepackage{upquote}}{}
\IfFileExists{microtype.sty}{% use microtype if available
  \usepackage[]{microtype}
  \UseMicrotypeSet[protrusion]{basicmath} % disable protrusion for tt fonts
}{}
\makeatletter
\@ifundefined{KOMAClassName}{% if non-KOMA class
  \IfFileExists{parskip.sty}{%
    \usepackage{parskip}
  }{% else
    \setlength{\parindent}{0pt}
    \setlength{\parskip}{6pt plus 2pt minus 1pt}}
}{% if KOMA class
  \KOMAoptions{parskip=half}}
\makeatother
\usepackage{xcolor}
\IfFileExists{xurl.sty}{\usepackage{xurl}}{} % add URL line breaks if available
\IfFileExists{bookmark.sty}{\usepackage{bookmark}}{\usepackage{hyperref}}
\hypersetup{
  pdftitle={Temel Bilgi Teknolojileri Kullanımı},
  pdfauthor={Msc.Ali Mertcan KÖSE},
  pdfborder={0 0 0},
  breaklinks=true}
\urlstyle{same}  % don't use monospace font for urls
\newif\ifbibliography
\usepackage{graphicx,grffile}
\makeatletter
\def\maxwidth{\ifdim\Gin@nat@width>\linewidth\linewidth\else\Gin@nat@width\fi}
\def\maxheight{\ifdim\Gin@nat@height>\textheight\textheight\else\Gin@nat@height\fi}
\makeatother
% Scale images if necessary, so that they will not overflow the page
% margins by default, and it is still possible to overwrite the defaults
% using explicit options in \includegraphics[width, height, ...]{}
\setkeys{Gin}{width=\maxwidth,height=\maxheight,keepaspectratio}
\setlength{\emergencystretch}{3em}  % prevent overfull lines
\providecommand{\tightlist}{%
  \setlength{\itemsep}{0pt}\setlength{\parskip}{0pt}}
\setcounter{secnumdepth}{5}

% set default figure placement to htbp
\makeatletter
\def\fps@figure{htbp}
\makeatother

\usepackage[document]{ragged2e}
\usepackage{caption}
\usepackage{graphicx}
\usepackage{fancyvrb}

\title{Temel Bilgi Teknolojileri Kullanımı}
\subtitle{Lecture 1}
\author{Msc.Ali Mertcan KÖSE}
\date{}
\institute{İstanbul Kent Üniversitesi}

\begin{document}
\frame{\titlepage}

\begin{frame}

\begin{block}{Outline}

\begin{enumerate}
\item
  Donanım

  \begin{itemize}
  \tightlist
  \item
    \emph{Kavramlar}
  \item
    \emph{Bilgisayar performansı}
  \item
    \emph{Bellek ve Depolama}
  \item
    \emph{Giriş, Çıkış Birimleri}
  \end{itemize}
\item
  Yazılım

  \begin{itemize}
  \tightlist
  \item
    \emph{Kavramlar}
  \end{itemize}
\item
  Ağ

  \begin{itemize}
  \tightlist
  \item
    \emph{Ağ Türleri}
  \item
    \emph{Veri Aktarımı}
  \end{itemize}
\end{enumerate}

\end{block}

\end{frame}

\hypertarget{donanux131m}{%
\section{Donanım}\label{donanux131m}}

\begin{frame}{Kavramlar}
\protect\hypertarget{kavramlar}{}

\justify

``Bilgi çağı'' ya da ``dijital çağ'' olarak adlandırılan 21. y.y.,
teknolojik yenilikler insanlığın hizmetine sunmakla kalmamış, farklı
disiplinerde yeni bilimsel yöntemlerin gelişmesine de imkan tanımıştır.

Bilgi?

\begin{itemize}
\tightlist
\item
  \textbf{Bilgi:} Genel olarak ve ilk sezi durumunda zihnin kavradığı
  temel düşünceler(TDK).
\end{itemize}

Dijital nedir?

\begin{itemize}
\tightlist
\item
  \textbf{Dijital:} Verilerin bir ekran üzerinde elektronik olarak
  gösterilmesi (TDK).
\end{itemize}

\end{frame}

\begin{frame}{Kavramlar}
\protect\hypertarget{kavramlar-1}{}

\justify

\textbf{Veri:} Bilgisayara girilen işlenmemiş durumundaki bilgilerdir.
Yani veri belirli konulardaki gerçeklerin sembolik ifadesidir.

\textbf{Enformasyon:} Verinini ilişkili bağlantılar sonucunda anlam
kazanmış halidir.

\textbf{Bilgi:} Bilen tarafından içselleştirildiği, tecrübe ve algıları
tarafından şekillendirildiği için genellikle kişisel ve özneldir.

\textbf{Bilgelik:} Şuana kadar üzerinde durduğumuz veri, enformasyon ve
bilgi tamamlanmış süreçler sonunda ulaştığımız durumlar olarak
değerlendirilebilir. Diğer Anlamıyla Bilgelik Bilgi piramidinin son
basamağı olmakla beraber bilgi üzerinden yeni fikir üretme veya buluş
aşamasıdır.

\end{frame}

\begin{frame}{Kavramlar}
\protect\hypertarget{kavramlar-2}{}

\begin{figure}
\centering
\includegraphics[width=0.5\textwidth,height=\textheight]{C:/Users/Ali/Desktop/sunum/sekil1.jpeg}
\caption{Bilgi Piramidi.}
\end{figure}

\end{frame}

\begin{frame}{Bilgisayar Performansı}
\protect\hypertarget{bilgisayar-performansux131}{}

\justify

\begin{quote}
Bilgisayar(Computer), uzun ve çok karmaşık hesapları bile büyük bir
hızla yapabilen, mantıksal(logic) bağlantılara dayalı karar verip işlem
yürüten makinedir. Kısacası, bilgisayar, bilgi işleyen elektronik bir
aygıttır.
\end{quote}

\end{frame}

\begin{frame}{Bilgisayar Performansı}
\protect\hypertarget{bilgisayar-performansux131-1}{}

\justify

Bugünkü bilgisayarın ve buna bağlı teknolojilerin ortaya çıkması tek bir
dönemde olmamaıştır. Tarihin her bir döneminde ortaya çıkan gelişmeler,
araştırmalar ve buluşlar, bugünkü bilgisayarın ortaya çıkmasında birer
adım olmuştur. Bugün kullandığımız bilgisayarın oluşmasında yer alan
gelişmelerin kronolojik sıralamasını incelersek; M.Ö. 3 bin yıllarında
paralı alışverişin başlaması ile sayma işlemine ihtiyaç duyuldu. ilk
zamanlarda 10 sayısına kadar saymak için parmakalr kullanıldı. Bu yöntem
yetmeyince de, M.Ö. 2600 yıllarında bu işlemlerde kullanılmak üzere
\textbf{abaküs} adı verilenn hesaplayıcılar bulundu.

\end{frame}

\begin{frame}{Bilgisayar Performansı}
\protect\hypertarget{bilgisayar-performansux131-2}{}

\begin{figure}
\centering
\includegraphics[width=0.7\textwidth,height=\textheight]{C:/Users/Ali/Desktop/sunum/kis2.jpg}
\caption{Soldan sağa doğru; Blaise Pascal, Charles Babbage, Alan Turing
ve Howard Aiken.}
\end{figure}

\justify

Heidelberg Üniversitesinde Wilhelm Shickard (1624), 4 fonksiyonlu hesap
makinesini yaptı. Pariste Blaise Pascal(1642), ilk nümerik hesaplama
makinesini yaptı. İngiliz matematikçi Charles Babbage(1812) fark
makinesi üzerinde çalışmalarda bulundu ve bu çalışmalar ile bilgisayarın
temelinin atıldığı söylenebilir.

\end{frame}

\begin{frame}{Bilgisayar Performansı}
\protect\hypertarget{bilgisayar-performansux131-3}{}

\justify

Alan Turing (1936) modern bilgisayarların geliştirilmesinden çok önce,
20 y.y.'ın 2. Dünya savaşı öncersi yıllarında tasarladığı sanal makine
(\textbf{Turing Makinesi}) ile algoritma olarak tabir edilen her türlü
mantıksal işlem bütününü mekanik süreçlere indirgebilecek bir makine
ortaya koymuştur. IBM şirketi adına çalışma yapan Harvard Üniversitesi
Akademisyeni Howard Aiken ve Brown (1937)'un geliştirdiği Mark 1, ilk
defa olarak insan müdhalesi olmaksızın çalışan sayısal otamatik
bilgisayardır. Bu bilgisayar 4 işlemlerin yapıldığı ve bu işlemlerin
depo edildiği bir makine iken, John Atansoff ve Clifford Berry (1939)
ilk elektronik bilgisayarı icat etmiştir.

\end{frame}

\begin{frame}{Bilgisayar Performansı}
\protect\hypertarget{bilgisayar-performansux131-4}{}

\begin{figure}
\centering
\includegraphics[width=0.9\textwidth,height=\textheight]{C:/Users/Ali/Desktop/sunum/kis3.jpg}
\caption{ENIAC(1945), UNIVAC(1960),IBM(1981) ve Macintosh(1984).}
\end{figure}

\end{frame}

\begin{frame}{Bellek ve Depolama}
\protect\hypertarget{bellek-ve-depolama}{}

\begin{quote}
\textbf{Donanım:} Bilgisayarın fiziksel ve elektronik yapısını oluşturan
ana birimlerin ve çevre birimlerin tümüne \emph{donanım} denir. Örnek
olarak; Ekran, klavye, ana kart, kablo gibi\ldots{}
\end{quote}

\end{frame}

\begin{frame}{Bellek ve Depolama}
\protect\hypertarget{bellek-ve-depolama-1}{}

\begin{figure}
\centering
\includegraphics[width=0.7\textwidth,height=\textheight]{C:/Users/Ali/Desktop/sunum/kis4.jpg}
\caption{Anakart.}
\end{figure}

\end{frame}

\begin{frame}{Bellek ve Depolama}
\protect\hypertarget{bellek-ve-depolama-2}{}

\justify

\textbf{Anakart:}Bilgisayara takılan bütün parçalar arasındaki
bağlantıyı bu kart sağlar.

\textbf{Merkezi İşlem Birimi(CPU):} Bilgisayarın transistöründen
yongasında ve diğer bütün parçalarına kadar bütün birimlerinin yönetim
işlemlerini CPU yerine getirir.

\textbf{Ana Bellek(RAM):} Açılan kullanıcı dosyalarını ve program
dosyalarını geçici olarak tutar ve burada dosyalarla ilgili işlemlerin
yapılmasında olanak verir.

\end{frame}

\begin{frame}{Bellek ve Depolama}
\protect\hypertarget{bellek-ve-depolama-3}{}

\justify

\textbf{Yalnız Okunabilir Bellek (ROM) ve Bios:} Elektriğin kesilmesi
veya bilgisayarın kapatılması durumunda kaybolmayacak sistem bilgileri
ve BİOS(Bilgisayarın açılarak çalışır duruma gelmesini sağlayan küçük
bir kontrol grubudur) bu Rom bellek Chip'ine yerleştirilmiştir.

\textbf{Ön Bellek:} Ram Belliğin hızı CPU hızından düşüktür. bu yüzden
ön bellek Ram ile CPU arasındaki veri transferlerinde görev yapar.

\textbf{Ekran Kartı:} Ram Bellekteki bilgilerin(açık olan dosyaların)
görüntülerinin ekrana iletişmesini sağlar.

\end{frame}

\begin{frame}{Bellek ve Depolama}
\protect\hypertarget{bellek-ve-depolama-4}{}

\begin{figure}
\centering
\includegraphics[width=0.7\textwidth,height=\textheight]{C:/Users/Ali/Desktop/sunum/kis6.jpg}
\caption{HDD ve SSD}
\end{figure}

\end{frame}

\begin{frame}{Bellek ve Depolama}
\protect\hypertarget{bellek-ve-depolama-5}{}

\justify

\textbf{Sabit Disk:} Bilgi depolamak amacı ile kullanılan kasa
içerisindeki kutuya yerleştirilmiş ikincil bellek birimidir.
Bilgisayardaki yazılımlar genellikle sabit diske kurulur. Kalıcı olarak
saklanması istanene veri sabit diske kaydedilir. Masaüstü bilgisayarlara
SATA, USB ve SAS kabloları ile bağlanan modelleri vardır. SSD'ler
performans, elektrik tüketimi, ses ve dayanıklılık açısından manyetik
disklere göre daha iyi olmalarına karşın kapasiteleri daha düşük ve
fiyatları daha yüksektir.

\end{frame}

\begin{frame}{Bellek ve Depolama}
\protect\hypertarget{bellek-ve-depolama-6}{}

Table 1. Boyut Kavramları

\begin{table}[H]
\centering
\begin{tabular}{l|l}
\hline
\textbf{Sayi} & \textbf{Boyut}\\
\hline
0 veya 1 & 1 bit\\
\hline
8 bit & 1 byte\\
\hline
1024 byte & 1 kilobyte\\
\hline
1024 KB & 1 megabyte\\
\hline
1024 MB & 1 gigabyte\\
\hline
1024 GB & 1 terabyte\\
\hline
\end{tabular}
\end{table}

\end{frame}

\begin{frame}{Giriş ve Çıkış Birimleri}
\protect\hypertarget{giriux15f-ve-uxe7ux131kux131ux15f-birimleri}{}

\begin{figure}
\centering
\includegraphics[width=0.7\textwidth,height=\textheight]{C:/Users/Ali/Desktop/sunum/kis7.jpg}
\caption{Giriş-İşlem Birimi-Çıkış Ünitesi Diyagramı}
\end{figure}

\end{frame}

\begin{frame}{Giriş ve Çıkış Birimleri}
\protect\hypertarget{giriux15f-ve-uxe7ux131kux131ux15f-birimleri-1}{}

\begin{figure}
\centering
\includegraphics[width=0.7\textwidth,height=\textheight]{C:/Users/Ali/Desktop/sunum/kis8.jpg}
\caption{Giriş- Çıkış Birimleri}
\end{figure}

\emph{Kişisel bilgisayarlar bip sesi çıkarırken farklı seslerin elde
edilerek çıkış birimlerine aktarılması için Ses kartının yer alması
gerekir.}

\end{frame}

\hypertarget{yazux131lux131m}{%
\section{Yazılım}\label{yazux131lux131m}}

\begin{frame}{Temel kavramlar}
\protect\hypertarget{temel-kavramlar}{}

\justify

\begin{quote}
\textbf{Yazılım:} Bilgisayarı çalıştırmaya yarayan, fiziksel kısım
dışında kalan her şeye yazılım denir. Yazılım programları ifade eder.
Bilgisayar bu programlar ile istenildiği gibi yönlendirilir.
\end{quote}

\end{frame}

\begin{frame}{Temel kavramlar}
\protect\hypertarget{temel-kavramlar-1}{}

\begin{itemize}
\tightlist
\item
  İşletim Sistemi Yazılımları

  \begin{itemize}
  \tightlist
  \item
    Windows
  \item
    Macos
  \item
    Linux
  \item
    Android
  \item
    IOS
  \end{itemize}
\item
  Uygulama Yazılımlar

  \begin{itemize}
  \tightlist
  \item
    Microsoft Office
  \item
    Skype
  \item
    Google Chrome
  \item
    Notepad
  \item
    Skype vb.
  \end{itemize}
\end{itemize}

\end{frame}

\begin{frame}{Temel kavramlar}
\protect\hypertarget{temel-kavramlar-2}{}

\justify

\textbf{Program:} Belirli amaca yönelik olarak yazılmış bilgisayarda
yerine getirilmesi istenilen sıralı işlemler için komut ve işlem
adımlarının tümüne program denmektedir.

\textbf{Program Dili:} Bilgisayar programlar yazımında belli kurallar
dizisi takip edilir. Bu kurallar dizisinin oluşturduğu gruba
\emph{Bilgisayar Programlama Dili} denir.

\textbf{Makine Dili:} Makine tarafından hiçbir değişikliğe uğratılmadan
kullanabilen dillerdir.

\textbf{Sembolik Dili:} Sembollerle yazılan programa dilidir.

\end{frame}

\begin{frame}{Temel kavramlar}
\protect\hypertarget{temel-kavramlar-3}{}

\justify

\textbf{Alt düzey Programlama Dilleri:} Makine diline çok yakındaır.
Yazılan programlar küçük bir çevirme işlemi ile makine diline
dönüştürülür.

\textbf{Üst Düzey Programlama Dilleri:} Bilgisayar kullanıcısının
kolaylıkla yazabildiği makine diline mutlaka çevrilmesi gerek dillerdir.

\end{frame}

\begin{frame}{Temel kavramlar}
\protect\hypertarget{temel-kavramlar-4}{}

\begin{figure}
\centering
\includegraphics[width=0.5\textwidth,height=\textheight]{C:/Users/Ali/Desktop/sunum/kis9.jpg}
\caption{En Çok Kullanılan Program Dİlleri}
\end{figure}

\end{frame}

\hypertarget{aux11flar}{%
\section{Ağlar}\label{aux11flar}}

\begin{quote}
Bilgisayar ağı: En az iki bilgisayarın kablolu ve kablosuz olarak
birbiri ile bağlantı oluşturmasıdır. Bu bağlantı sonucunda iletişim ile
veri alışverişi yapılabilir.
\end{quote}

\begin{frame}{Ağ Türleri}
\protect\hypertarget{aux11f-tuxfcrleri}{}

\begin{itemize}
\item
  \textbf{LAN(Local Area Network):} Yerel bölge ağları, bilgisayarları
  ve diğer bilgi işlem araçlarını kısıtlı bir fiziksel çevre dahilinde
  birbirine kablolu olarak bağlar(ofis,ev,bina gibi).
\item
  \textbf{WLAN(Wireless Local Area Network):} Kablosuz yerel ağlar, dar
  coğrafik alanda firmaların veya hanehalkarlın kablosuz olarak
  bağlantılı olması durumudur.
\item
  \textbf{WAN(Wide Area Network):} Geniş bölge ağlar, ülke ve eyalet
  gibi geniş coğrafi bölgelerde kullanıalabilecek elektronik haberleşme
  ağıdır.
\end{itemize}

\end{frame}

\begin{frame}{Ağ Türleri}
\protect\hypertarget{aux11f-tuxfcrleri-1}{}

\justify

\textbf{İnternet:} Bilgisayar ve Bilgisayar ağlarının arasındaki
iletişimin evrensel sisteminden oluşur, Bu iletişim TCP/IP protocols
yardımıyla sağlanır. İnternet başkarda basit veri alışverişini sağlamak
amacıyla ortaya çıkmasına rağmen bugünlerde bütün toplulukların sosyal
alanlarını(Ekonomi, Sosyal platform, Bilgi edinme, Sağlık Eğitim gibi)
etkilemiştir.

\textbf{Extranet:} İnternetin bir bölümü olarak bağımsız işbirlikçilerin
eriştiği bir ağdır.

\end{frame}

\begin{frame}{Veri Aktarımı}
\protect\hypertarget{veri-aktarux131mux131}{}

\justify

\textbf{İndirme(Download):} Yerel bilgisayar üzerinde ağ bilgisayarından
dijital veri kopyası almak olarak ifade edilen terime
\emph{İndirme(Download)} denir. Diğer taraftan ağ üzerinde dijital
içerik yerleştirmeye \emph{Yükleme(Upload)} denir.

\textbf{Bitrate:} Modem(ağ) vesilesi ile veri transferi yapılması
sonucunda ağ aracılığıyla dijital veri akışı hızı için ölçüm birimine
\emph{Bitrate} denir.

1,000 bit/s rate = 1 kbit (her saniye üzerinden)

1,000,000 bit/s rate = 1 Mbit

1,000,000,000 bit/s rate = 1 Gbit

\end{frame}

\begin{frame}{Veri Aktarımı}
\protect\hypertarget{veri-aktarux131mux131-1}{}

\begin{itemize}
\tightlist
\item
  \textbf{Bağlantı Metodları}

  \begin{itemize}
  \tightlist
  \item
    Mobil
  \item
    Uydu
  \item
    Wireless(Wi-Fi)
  \item
    Kablo
  \item
    Geniş bant
  \end{itemize}
\end{itemize}

\end{frame}

\end{document}
